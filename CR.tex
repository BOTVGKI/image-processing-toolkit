\documentclass[12pt,a4paper,notitlepage]{report}
%\usepackage[showframe=false]{geometry}
%\usepackage[francais]{babel}
\usepackage[utf8]{inputenc}
%\usepackage{kpfonts}
\usepackage{amsmath}
\usepackage{amsfonts}
\usepackage{mathtools}
%\usepackage{listings}
\usepackage[showframe=false,tmargin=2.5cm,bmargin=2.5cm,lmargin=2.5cm,rmargin=2.5cm]{geometry}
%\usepackage[T1]{fontenc}

%\lstset{
%  basicstyle=\ttfamily,
%  mathescape
%}
\begin{document}
\begin{Large}
\textbf{Mohamed Attia}\\
\indent\textbf{Bacar Said}\\

M2-MMSI
\end{Large}

\begin{center}
\begin{huge}
\vspace{6cm}
\textbf{Compte rendu des TP}\\
\vspace{1cm}

Introduction au traitement d'images
\end{huge}

\begin{LARGE}

\vspace{0.5cm} MA0944
\end{LARGE}

\vspace{8cm}
\begin{large}
15 Février 2018

\end{large}
\end{center}

\newpage
Le but de ce travail est de programmer quelque méthode de traitement d’image sur le logiciel Matlab/Octave. Ainsi nous avons programmer un certain nombre de fonctions qu’on a classé sur deux dossiers différents. (Tous nos fichier Matlab ‘*.m ‘ sont à utiliser comme des script).
\\
Les codes qu’on a créés ici consiste à prendre une image et faire varier le(s) niveau(x) celons des formules bien précis existant.
Voici les noms de différents fonction programmer sur Matlab
	\includegraphics[scale=0.4]{test.png}

\begin{itemize}
	\item RGBTOGRAY1, RGBTOGRAY2, RGBTOGRAY3\\
	\includegraphics[scale=0.35]{1.png} 
\includegraphics[scale=0.35]{2.png}
\includegraphics[scale=0.34]{3.png}
	\item RGBTOYIQ, YIQTORGB\\
	\begin{center}
		\includegraphics[scale=0.35]{4.png} 
		\includegraphics[scale=0.35]{5.png}
	\end{center}
	\item RGBTOYUV, YUVTORGB
	\begin{center}
		\includegraphics[scale=0.35]{6.png} 
		\includegraphics[scale=0.4]{7.png}
	\end{center}
	\item RGBTOHSV, HSVTORGB
	\begin{center}
		\includegraphics[scale=0.35]{8.png} 
		\includegraphics[scale=0.4]{9.png}
	\end{center}
	\item RGBTOI1I2I3, I1I2I3TORGB
	\begin{center}
		\includegraphics[scale=0.35]{10.png} 
		\includegraphics[scale=0.35]{11.png}
	\end{center}
	\item RGBTONRGB1, RGBTONRGB2
	\begin{center}
		\includegraphics[scale=0.35]{12.png} 
		\includegraphics[scale=0.35]{13.png}
	\end{center}
	\item Histogramme
	\begin{center}
		\includegraphics[scale=0.35]{14.png} 
	\end{center}
	\item ImageIndexe,
	\item Histo3D
\end{itemize}

\subsection*{Seuillage:}
Le dossier seuillage contient plusieurs scriptes qui prend une image (noir en blanc ou image couleur) puis il la binarise où fait une segmentation à partir d’une seuille donnée pas l’utilisateur.
\begin{itemize}
	\item SeilManuel\\
		\begin{center}
	\includegraphics[scale=0.35]{manuel.png} \\
	Seuil=100
		\end{center}
	\item Seuilauto
	\item Seuillagemoy
		\begin{center}
	\includegraphics[scale=0.35]{moyenne.png} \\
		\end{center}
	\item seuillegemed
			\begin{center}
	\includegraphics[scale=0.35]{mediane.png} \\
		\end{center}
	\item Otsou1, Otsou2,
	\begin{center}
		\includegraphics[scale=0.35]{otsu.png} 
		\includegraphics[scale=0.4]{7.png}
	\end{center}
	\item isodata
	\item Berson
	\item Nilback
	\item Sauvola
\end{itemize}

\subsection*{Validation des scripts:}

\subsection*{Conclusion:}
Ce TP nous a permis d’acquérir des nombreuse fonction de transformation d’image. 


\end{document}